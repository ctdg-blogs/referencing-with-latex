\documentclass[11pt]{article}
\usepackage{setspace}
\usepackage[T1]{fontenc}
\usepackage[utf8]{inputenc}
\usepackage{lmodern}
\usepackage{times}
\usepackage[notes, backend=biber]{biblatex-chicago}
\bibliography{reference}

\author{516370910214 Sun,Yi}
\title{Foreign Presence in Early Modern China: A Lack of Compassion and Communication}

\begin{document}
\maketitle
\begin{spacing}{2.0}
\paragraph{}
As a common knowledge in China, the country encountered with many foreign enemies in early modern, but foreigners should be by no means "purely ill". The canvas of foreign presence in China is painted with far less enough communications and shallow comprehension. We will further look into the circumstances, under which foreign diplomats, residents, and missionaries live and work, trying to figure out their excellency, deficiency, and the response of Chinese natives to these behaviors.
\\
\textbf{I.G. and IMCS}

In early modern, many South East Asia countries were at the mercy of Western syndicates or families. 
East India Company, the British representative of "international trade and Opium trafficking", encountered with the Indian Rebellion in 1857.\autocite{The-Opium-War}
The family of General Douglas MacArthur also monopolized industry, agriculture, and even army in Philippines, yet was commemorated by local people for his leadership in Philippines' anti-Japanese war. 
It is hard to tell whether these financial groups were purely good or bad. 
And the situation is alike in China, even though "its sovereignty was impaired but never vanquished" due to the huge territory and that foreign powers were always at odds with the division of shares\autocite{Foreign-Presence-in-China}. 

Among the foreign presence in Chinese authority in early 20th century, the Imperial Maritime Custom Service(IMCS) is the most representative and profitable one. Named as Chinese governmental tax collecting agency, the Custom was indeed established by foreign consuls in Shanghai in 1854, so as to collect maritime trade taxes during the Taiping Uprising, in which time Chinese officials cannot impose tarrif by themselves. The creation of IMCS was largely due to Sir Robert Hart's good relationship with Prince Kung(Yixin).\autocite{Robert-Hart} This first Inspector-General (I.G.), unlike some of his successors, acknowledged his identity as an employee of Chinese government over decades. Later, it developed branches such as customs and postal administration, harbor and waterway management, weather reporting, and anti-smuggling operations. Foreign officials even map, lit, policed the China coast and the Yangtze River, published custom service related newspaper. The departments, designed to serve local foreigners, also contributed to Imperialism information service and paying off loans imposed on China.

Foreign officials of diplomatic bodies were supposed to protect only the profit of their own countries. Note that the imperialism powers obtained their majority profit by spreading opium and colonizing, they are intentionally invading China's sovereignty for their own profit and collecting taxes for foreign governments. Due to the inability of Qing court, foreign diplomats even need not to communicate with or respect Chinese officials. And thus few foreigners, such as Robert Hart, would conduct diplomatic affairs with China in a courteous manner. The short-sightedness of both sides resulted in the mounting anti-Imperialism atmosphere among Chinese ordinary people. We also have to acknowledge that, IMCS somehow protected the tax from being wasted by Qing Empire and introduced maritime custom and police system to China.
\\
\textbf{Leased Territory Residents}

In the summer of 1900, the Boxer Rebellion broke out as a result of the anti-foreigner atmosphere among native Chinese. The diplomats, still in no awareness of the inefficiency of their common Western-style international public relation handling methods, called on the Eight-Nation Military Alliance to force China government to sign the Boxer Protocol, keeping their citizens under the protection of leased territories, where China's sovereignty was explicitly extinguished. 

The extraterritoriality of leaseholds was a combination of over-action to Boxer Rebellion and pursuit of maximized interest. Foreigners were preferred to make loans, set up companies, and operating Chinese infrastructures. They were also free from Chinese government's restricts, regulations and taxes. Gunboats and armed forces were also permanent and more sizable after the Protocol, which did protected the personal safety as well as fortune of foreign bank owners, missionaries and press corps, yet in an aggressive way such that it is overwhelmingly remembered as a notorious invasion of Chinese territory among local people.

The foreign life, from then on, was a closed Western-style circle. Shanghai leaseholds, for example, looked like a British city with abundant garden, tennis court, and race courses. Residents enjoyed their press, afternoon teas, and night outs. Like any other self-sufficient society, they divided classes according to type of business, clubs belonged, and number of ponies. Local Chinese were of course at the bottom of this Western class pyramid.

It is harshly criticized that foreign residents in Shanghai set a "Dogs and Chinese not Admitted" sign on Huangpu Park. Although the existence of signboard was vague, and even the whole story came, to some extent, from a misunderstanding of the parks' regulation, the arrogance of foreign residents and corresponding dissatisfactory of Chinese answered to the description\autocite{Dogs-and-Chinese-not-Admitted}. In general, without any remarkable endeavor to solve the breach, both foreigners and Chinese chose to live a separated self-contained life, only dealing with each other on some similar and trivial topics. For example, there were Chinese Amahs served as baby-nursers in foreign families.\autocite{Out-of-China}
\\
\textbf{Christian Missionaries}

Missionaries travelled to China to preach. Although most of them were assigned by churches, they were almost the only foreign groups seeking deep communication with native people, struggling to help them to reform China into a modern country. It is generally acknowledged that the Protestants may focus on a relatively broadened goal, ranging from the conversion of individuals to Christianizing Chinese society, while Romantic Catholic missionaries would focus on saving individual souls. Yet their achievements turned out to be generally the same, i.e., both denominations focused on converting Chinese natives, performing Western medical treatments, and popularizing reformed education.

The enthusiasm of Western missionaries were soon cooled down by indifferent local people. The Chinese had a history with no monotheism except for the fear of emperors. At the very beginning, the clergymen had to concentrate in rural areas, buying farm products, offering candies, and even treat patients so as to attract audience for their sermon\autocite{finding-missionaries}.
The hardship was alleviated after the First Sino-Japanese War (War of Jiawu), when China society gradually noticed the importance of learning Western technology as well as regulations. Missionaries soon joined the reform currents of China. They took part in education for women (Ginling College in Nanking in 1915) and Anti-footbinding movement, paid attention to urban and labour problems(YMCA $\&$ YWCA), introduced Western public health systems. The foundation of Saint John's University and Dingli Hospital brought forth reputations to missionaries and their Christian church. This time, it was the Chinese, in turn, turned their back to the foreigners and sticked to savage critics and sabotage towards those so-called strangers.

In despite of the increasing number of missionaries, communicants, and Chinese converts receiving educational and medical services, the missionaries, to some extent, suffered from the stereotype of foreign invaders and representatives of Imperialism. Their work had eventually no big change after the national Anti-Christian movement in 1920s.
\\
\textbf{conclusion}

For decades, foreigners in early 20th century were regarded as the spokesmen of "vicious Imperialism" in China. And Chinese were stigmatized as "sick man of East Asia" in return. The stereotype came from hundred years of separation of Western and Eastern world, and was worsen by several wars during this era. Even the later propaganda of World War II did not change this mutual discrimination towards each other. With the globalization of modern society and escalation of available biography as well as detailed documents, people from both China and Western countries are more likely to understand the mis-function of communication and the consequently conflicts. Lessons have taught to both sides that different nations should cooperate with and reconcile to each other sincerely, so as not to level up disputes into a war.
\end{spacing}

\printbibliography
\end{document}